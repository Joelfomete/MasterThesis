\abstractintoc % Add abstract to Table of Contents  
\abstractnum   % Format abstract like a chapter
               % Remove if abstract should not be on its own page


\begin{abstract}
Life expectancies at birth are usually computed from period life tables, i.e. on the basis of mortality in a particular calendar year. Over the last century, there has been a worldwide decline in mortality rates at most ages leading to an increase of the life expectancy. But comparing countries based on the life expectancy from period life tables may ignore different historical mortality developments in those countries and therefore lead to a wrong conclusion. Consequently, instead of comparing life expectancy for countries based on period life tables, it may be more appropriate do such comparison based on life expectancy from cohort life tables. Since cohort life expectancies can only be obtained for older cohorts i.e. those born more than a hundred years ago, in this thesis we suggest that for younger cohorts one may consider the expected number of years lost up to a given age.
When we consider life expectancy based on period mortality, one finds that since the 80's, women in Spain and Italy have had higher life expectancy than those in Norway and Sweden. However, if we consider the expected number of years lost for different cohorts in Spain, Italy, Norway and Sweden, we observe that women in Scandinavia are still expected to lose fewer years, i.e. live longer, than those in the Mediterranean. The results indicated by period data may be due to a selection effect and may therefore be an artifact.

\end{abstract}