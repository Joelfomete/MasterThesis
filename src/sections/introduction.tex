\chapter{INTRODUCTION}
\label{sec:intro}

According to data from Statistics Norway, life expectancy at birth for Norwegian and Swedish women in 2017 was respectively 84.3 years and 84.1 years, while in Italy and Spain it was respectively 85.2 years and 86.1 years. But about 60 years earlier (in 1960), women in Scandinavia had higher life expectancy than women in the Mediterranean ( Norway 76.0 years, Sweden 74.9 years, Italy 72.3 years and Spain 72.2 years). Thus, there has been a major change in life expectancy at birth over the last fifty years. \parencite{S19}

Life expectancies at birth are usually computed from period life tables, ie. on the basis of mortality in a particular calendar year. When there are major changes in mortality, and these changes occur at different times in different countries, life expectancy calculated based on period life tables can give a misleading picture of life expectancy in different countries. It would then be better to calculate life expectancy based on a cohort, ie. from the mortality rates of persons who were born in the same year. However, a problem with the latter approach is that one cannot find the life expectancy of a cohort until the entire cohort has died out. A partial solution is to compute the expected number of years lost for the cohorts up to given ages. \parencite{BK19}

As a cohort of people ages, the individuals at highest risk tend to die first. This differential selection can produce patterns of mortality for the population as a whole that are surprisingly different from the patterns for sub-populations or individuals. This can be illustrated by the use of frailty models. (\cite{AOH08})

The data we will use are from the Human Mortality Database (HMD) which contains a wealth of information about detailed mortality and population data for about 40 countries. The input data consist of death counts from vital statistics, plus census counts, birth counts, and population estimates from various sources. Detailed information about mortality rates for different countries can reveal information about changes in life expectancy.

This thesis is organized into five main chapters. We will start in the second chapter by considering some basic concepts and results in survival analysis and also consider parametric inference for survival data.
The third chapter contains information about Lexis diagram and how it is constructed. And then we will see how life line, age, period and cohort data are visualize in a Lexis diagram. We will define mortality rate and show how the exposure and the number of deaths are computed from the Lexis triangles. These are then used to compute the mortality hazard for period data and cohort data for Norwegian females from the calendar year 1900 to the calendar year 2014. We will plot the period mortality rates for Norwegian females during five different periods and similarly for the cohort mortality rates. We define and compute the life expectancy. We then plot the life expectancy at birth for period data. We also plot the life expectancy up to age $a$ first as a function of cohort and then as a function age for different choices of $a$ for Norwegian females. We end the second section by computing and plotting the expected number of years lost up to a certain age $a$ for different cohorts of Norwegian females.

In the fourth chapter we compare the mortality for two countries in the Mediterranean and two countries in Scandinavia. We begin by computing and plotting the mortality rates for Italy, Spain, Norway and Sweden.To have uniform data sets for the four countries we choose to look at developments from the calendar year 1910 to the calendar year 2014.The ages varies between 0 and 110 years. We then compare the evolution of mortality in those countries over the years. The mortality rates are again use to compute and plot the period life expectancy for the years 1920, 1950, 1980 and 2014. We also compute the expected number of years lost where we first fix the age and consider the expected number of years lost as a function of cohort and then we fix the cohort and look at the expected number of years lost as a function of age.

The fifth  chapter of the thesis focuses mainly on the frailty model. In the sixth chapter, we use the two-points frailty distribution with a baseline death intensity of the Gompertz-Makeham form to illustrate the difference between Scandinavia and the Mediterranean and discuss a possible explanation of why period data and cohort data may lead to different results.


% https://link.springer.com/article/10.1007%2Fs10433-013-0274-8